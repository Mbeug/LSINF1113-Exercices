\subsection{Question 1}

Nous allons donc vérifier les 3 propriétés :

\begin{enumerate}
    \item $||A + B|| \leq ||A|| + ||B||$
    \item $||A \cdot W|| \leq ||A|| \cdot ||W||$
    \item $||A \cdot B|| \leq ||A|| \cdot ||B||$
\end{enumerate}

\paragraph{1e propriété}

\begin{equation}
	\begin{aligned}
		||A + B|| &= max_{||V||=1}||(A + B)V||\\
		&= max_{||V||=1}||AV + BV||\\
		&= max_{||V||=1}||AV|| + max_{||V||=1}||BV||\\
	\end{aligned}
\end{equation}

On peut passer de la deuxième à la troisième ligne car quand on multiplie une matrice par un vecteur on obtient un vecteur, et on peut donc appliquer les propriétés des vecteurs.

D'autre part on a que :

\begin{equation}
	\begin{aligned}
		||A|| + ||B|| &= max_{||V||=1}||AV|| + max_{||V||=1}||AV||
	\end{aligned}
\end{equation}

\paragraph{2e propriété}

Pour le cas ou $W = 0$, on sait que multiplier une matrice par un vecteur nul donne un vecteur nul, et que la norme d'un vecteur nul est nul.

\begin{equation}
	\begin{aligned}
		||A \cdot W|| &= ||A \cdot 0||\\
		&= ||0||\\
		&= 0
	\end{aligned}
\end{equation}


\begin{equation}
	\begin{aligned}
		||A|| \cdot ||W|| &= ||A|| \cdot ||0||\\
		&= ||A|| \cdot ||0||\\
		&= 0
	\end{aligned}
\end{equation}

Pour le cas ou $W \neq 0$ 

\paragraph{3e propriété}

// TO DO

\subsection{Question 2}

Trouvons juste un contre exemple.

\begin{equation}
	A = 
	\begin{pmatrix}
		1 & 2\\
		3 & 4
	\end{pmatrix};\\
	B = 
	\begin{pmatrix}
		5 & 6\\
		7 & 8
	\end{pmatrix};\\
	C = A \cdot B = 
	\begin{pmatrix}
		19 & 22\\
		43 & 50
	\end{pmatrix}
\end{equation}

\begin{equation}
	\begin{aligned}
		||A \cdot B|| &= ||C|| = max_{i,j}|c_{i,j}| = 50\\
		||A|| \cdot ||B|| &= max_{i,j}|a_{i,j}| \cdot max_{i,j}|b_{i,j}| = 4 \cdot 8 = 32\\
	\end{aligned}
\end{equation}

Pour que la norme soit sub-multiplicative il faut que $||A \cdot B|| \leq ||A|| \cdot ||B||$. Or 50 n'est pas plus petit ou égal à 32, donc cette norme ne peut être sub-multiplicative.

\subsection{Question 3}

Pour cette question toutes les matrices inverses ont été calculées via \href{http://matrixcalc.org/en/}{Matrixcalc.org}. Pour des matrices 2x2 on peut aisément le faire "à la main" avec la formule suivantes :

\begin{equation}
	\begin{aligned}
		A = 
		\begin{pmatrix}
			a & b\\
			c & d
		\end{pmatrix}; A^{-1} = 
		\begin{pmatrix}
			a & b\\
			c & d
		\end{pmatrix}^{(-1)} = 
		\frac{1}{ad-bc}
		\begin{pmatrix}
			d & -b\\
			-c & a
		\end{pmatrix}
	\end{aligned}
\end{equation}

La condition pour appliquer cette méthode étant que $ad - bc \neq 0$ (sinon on divise par zéro, et ça on ne peut pas faire).

\subsubsection{Point 1}

\begin{equation}
	\begin{aligned}
		Cond (A) &= ||A|| \cdot ||A^{-1}||\\
		||\cdot||_1 &= max_{1 \leq j \leq n} \sum_{i=1}^m|a_{ij}|\\
		\begin{pmatrix}
			2 & 3\\
			3 & 4
		\end{pmatrix}^{(-1)} = 
		\begin{pmatrix}
			-4 & 3\\
			3 & -2
		\end{pmatrix}\\
			||A|| &= 7\\
			||A||^{(-1)} &= 7\\
			Cond (A) &= 7 \cdot 7 = 49
	\end{aligned}
\end{equation}

\subsubsection{Point 2}

\begin{equation}
	\begin{aligned}
		\begin{pmatrix}
			2 & 2.2\\
			3 & 3.2
		\end{pmatrix}^{(-1)} = 
		\begin{pmatrix}
			-16 & 11\\
			15 & -10
		\end{pmatrix}\\
		||A|| &= 5.4\\
		||A||^{(-1)} &= 31\\
		Cond (A) &= ||A|| \cdot ||A^{-1}||\\
		&= 5.4 \cdot 31 = 167.4
	\end{aligned}
\end{equation}

On ne peut rien dire sur le condition number sans avoir calculé $A^{-1}$.

\subsubsection{Point 3}

\begin{equation}
	\begin{aligned}X = 
		\begin{pmatrix}
			2 & 2.2\\
			3 & 3.2
		\end{pmatrix}; Y = 
		\begin{pmatrix}
			3\\
			4
		\end{pmatrix}
	\end{aligned}
\end{equation}

// TO DO