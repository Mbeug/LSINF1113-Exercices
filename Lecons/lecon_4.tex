\subsection{Question 1}

\begin{equation}
	\begin{aligned}
		X^TXA &= X^TY\\
		\begin{pmatrix}
			1 & 1 & 1 & 1 & 1 & 1\\
			95 & 85 & 80 & 70 & 60 & 70
		\end{pmatrix}
		\cdot
		\begin{pmatrix}
			1 & 95 \\
			1 & 85 \\
			1 & 80 \\
			1 & 70 \\
			1 & 60 \\
			1 & 70 
		\end{pmatrix}
		\cdot
		\begin{pmatrix}
			a_1\\
			a_2
		\end{pmatrix}
		&=
		\begin{pmatrix}
			1 & 1 & 1 & 1 & 1 & 1\\
			95 & 85 & 80 & 70 & 60 & 70
		\end{pmatrix}
		\cdot
		\begin{pmatrix}
			85\\
			95\\
			70\\
			65\\
			70\\
			80
		\end{pmatrix}\\
		\begin{pmatrix}
			6 & 460\\
			460 & 36050
		\end{pmatrix}
		\cdot
		\begin{pmatrix}
			a_1\\
			a_2
		\end{pmatrix}
		&=
		\begin{pmatrix}
			465\\
			36100
		\end{pmatrix}\\
		\begin{pmatrix}
			1 & 36050/460\\
			6 & 460
		\end{pmatrix}
		\cdot
		\begin{pmatrix}
			a_1\\
			a_2
		\end{pmatrix}
		&=
		\begin{pmatrix}
			36100/460\\
			465
		\end{pmatrix}\\
		\begin{pmatrix}
			1 & 36050/460\\
			0 & 460 - 6 \cdot \frac{36050}{460}
		\end{pmatrix}
		\cdot
		\begin{pmatrix}
			a_1\\
			a_2
		\end{pmatrix}
		&=
		\begin{pmatrix}
			36100/460\\
			465-6 \cdot \frac{36100}{460}
		\end{pmatrix}\\
		\begin{pmatrix}
			1 & 36050/460\\
			0 & 1
		\end{pmatrix}
		\cdot
		\begin{pmatrix}
			a_1\\
			a_2
		\end{pmatrix}
		&=
		\begin{pmatrix}
			36100/460\\
			\frac{465-6 \cdot \frac{36100}{460}}{460 - 6 \cdot \frac{36050}{460}}
		\end{pmatrix}\\
		\begin{pmatrix}
			1 & 36050/460\\
			0 & 1
		\end{pmatrix}
		\cdot
		\begin{pmatrix}
			a_1\\
			a_2
		\end{pmatrix}
		&=
		\begin{pmatrix}
			36100/460\\
			\frac{27}{47}
		\end{pmatrix}\\
		\begin{pmatrix}
			1 & 0\\
			0 & 1
		\end{pmatrix}
		\cdot
		\begin{pmatrix}
			a_1\\
			a_2
		\end{pmatrix}
		&=
		\begin{pmatrix}
			36100/460-\frac{27}{47} \cdot 36050/460\\
			\frac{27}{47}
		\end{pmatrix}
	\end{aligned}
\end{equation}

\begin{equation}
	\begin{aligned}    
		a_1 &= 33,457\\
		a_2 &= 0.5745
	\end{aligned}
\end{equation}

\subsection{Question 2}

\subsubsection{Explicit normal equation for the model $y = a_1 + a_2x$}

\begin{equation}
	\begin{aligned}
		\begin{pmatrix}
			m & \displaystyle\sum_{i=1}^{m} x_i\\
			\displaystyle\sum_{i=1}^{m} x_i & \displaystyle\sum_{i=1}^{m} x_i^2
		\end{pmatrix}
		\cdot
		\begin{pmatrix}
			a_1\\
			a_2
		\end{pmatrix}=
		\begin{pmatrix}
			\displaystyle\sum_{i=1}^{m} y_i\\
			\displaystyle\sum_{i=1}^{m} x_iy_i
		\end{pmatrix}
	\end{aligned}
\end{equation}

\subsubsection{Explicit normal equation for the model $y = a_1 + a_2x + a_3x^2$}

\begin{equation}
	\begin{aligned}
		\begin{pmatrix}
			m & \displaystyle\sum_{i=1}^{m} x_i & \displaystyle\sum_{i=1}^{m} x_i^2\\
			\displaystyle\sum_{i=1}^{m} x_i & \displaystyle\sum_{i=1}^{m} x_i^2 & \displaystyle\sum_{i=1}^{m} x_i^3\\
			\displaystyle\sum_{i=1}^{m} x_i^2 & \displaystyle\sum_{i=1}^{m} x_i^3 & \displaystyle\sum_{i=1}^{m} x_i^4\\
		\end{pmatrix}
		\cdot
		\begin{pmatrix}
			a_1\\
			a_2\\
			a_3
		\end{pmatrix}=
		\begin{pmatrix}
			\displaystyle\sum_{i=1}^{m} y_i\\
			\displaystyle\sum_{i=1}^{m} x_iy_i\\
			\displaystyle\sum_{i=1}^{m} x_i^2y_i
		\end{pmatrix}
	\end{aligned}
\end{equation}


\subsection{Question 3}

\subsubsection{Point 1}

\begin{equation}
	\begin{aligned}
		A &= 
		\begin{pmatrix}
			25 & 15 & -5\\
			15 & 18 & 0\\
			-5 & 0 & 11\\
		\end{pmatrix};
		L_A = 
		\begin{pmatrix}
			5 & 0 & 0\\
			3 & 3 & 0\\
			-1 & 1 & 3\\
		\end{pmatrix}\\
		B &= 
		\begin{pmatrix}
			4 & 12 & -16\\
			12 & 37 & -43\\
			-16 & -43 & 98
		\end{pmatrix};
		L_B =
		\begin{pmatrix}
			2 & 0 & 0\\
			6 & 1 & 0\\
			-8 & 5 & 3
		\end{pmatrix}
	\end{aligned}
\end{equation}

Le fonctionnement de l'algorithme est développé en détail sur Wikipédia \footnote{https://fr.wikipedia.org/wiki/Factorisation\_de\_Cholesky\#Algorithme}.

\code{matrices}{CholeskyFactorization.java}

\subsubsection{Point 2}

\begin{equation}
	\begin{aligned}
		A=\begin{pmatrix}
		a & b\\
		c & d
		\end{pmatrix}
	\end{aligned}
\end{equation}

La matrice $A$ est positive définie si $Z^TAZ > 0$, pour un $Z$ non nul quelconque dans $R^n$.

\begin{equation}
	\begin{aligned}
		\begin{pmatrix}
		z_1 & z_2
		\end{pmatrix}\cdot
		\begin{pmatrix}
		a & b\\
		c & d
		\end{pmatrix}\cdot
		\begin{pmatrix}
		z_1\\
		z_2
		\end{pmatrix} &> 0\\
		\begin{pmatrix}
		z_1\cdot a + z_2\cdot c & z_1\cdot b + z_2\cdot d
		\end{pmatrix}\cdot
		\begin{pmatrix}
		z_1\\
		z_2
		\end{pmatrix} &> 0\\
		z_1^2\cdot a + z_1\cot z_2\cdot c + z_1 \cdot z_2 \cdot b + z_2^2\cdot d &> 0
	\end{aligned}
\end{equation}

Vu que $Z \neq 0$, $z_1$ et $z_2$ ne peuvent pas être nuls en même temps.

\begin{equation}
	\begin{aligned}
		z_1=0; z_2 \neq 0 \rightarrow &\quad z_2^2d>0\\
		&\quad d>0\\
		z_1\neq 0; z_2 = 0 \rightarrow &\quad z_1^2a>0\\
		&\quad a>0\\
	\end{aligned}
\end{equation}

$z_1^2$ et $z_2^2$ étant forcément plus grands que $0$ car ils sont non nuls et au carré, donc forcément positif.

\subsubsection{Point 3}

// TO DO