\subsection{Question 1}

\begin{equation}
	\begin{aligned}
		p(x) &= \sum_{i=1}^{3}f(x_i)\Phi_i(x\\
		f'(x) &= \sum_{i=1}^{3}f(x_i)\Phi'_i(x)\\
		\Phi_1(x) &= \frac{(x-x_2)(x-x_3)}{(x_1-x_2)(x_1-x_3)} = (x^2 - (x_2+x_3)x+x_2x_3)\frac{1}{(x_1-x_2)(x_1-x_3)}\\
		\Phi_2(x) &= \frac{(x-x_1)(x-x_3)}{(x_2-x_1)(x_2-x_3)}\\
		\Phi_3(x) &= \frac{(x-x_1)(x-x_2)}{(x_3-x_1)(x_3-x_2)}\\
		\Phi_1(x)' &= \frac{2x - (x_2+x_3)}{(x_1-x_2)(x_1-x_3)}\\
		\Phi_2(x)' &= \frac{2x - (x_1+x_3)}{(x_2-x_1)(x_2-x_3)}\\
		\Phi_3(x)' &= \frac{2x - (x_1+x_2)}{(x_3-x_2)(x_3-x_1)}\\
		f'(x) &= f(x_1)\frac{2x - (x_2+x_3)}{(x_1-x_2)(x_1-x_3)} + f(x_2)\frac{2x - (x_1+x_3)}{(x_2-x_1)(x_2-x_3)}+f(x_3)\frac{2x - (x_1+x_2)}{(x_3-x_2)(x_3-x_1)}
	\end{aligned}
\end{equation}

\begin{equation}
	\begin{aligned}
		Erreur &= \frac{f^{(3)}(\xi(x))}{3!}(\prod_{i=1}^{3}(x - x^{(i)}))'\\
		&= \frac{f^{(3)}(\xi(x))}{3!}((x-x_1)(x-x_2)(x-x_3))'\\
		&= \frac{f^{(3)}(\xi(x))}{3!}(x^3-(x_1+x_2+x_3)x^2 + (x_1x_2+(x_1-x_2)x_3)x-x_1x_2x_3)'\\
		&= \frac{f^{(3)}(\xi(x))}{3!}(3x^2 - 2(x_1+x_2+x_3)x + x_1x_2+(x_1+x_2)x_3)\\
	\end{aligned}
\end{equation}

Si l'on remplace $x_1$, $x_2$ et $x_3$ par $x_1 = x-h$, $x_2 = x$ et $x_3 = x + h$, ont obtient :

\begin{equation}
	\begin{aligned}
		f'(x) &= f(x-h)\frac{2x - (x+x+h)}{(x-h-x)(x-h-(x+h))} + f(x)\frac{2x - (x+h+x+h)}{(x-(x-h))(x-(x+h))}+f(x+h)\frac{2x - (x-h+x)}{(x+h-x)(x+h-(x-h))}\\
		&= f(x-h)\frac{-h)}{2h^2} + f(x+h)\frac{h}{2h^2}\\
		&= \frac{f(x+h)-f(x-h)}{2h}
	\end{aligned}
\end{equation}

L'interpolation devient alors identique à la méthode du "two sided centered differencing'' vue au cours.

\subsection{Question 2}

\subsubsection{Point 1}

\begin{equation}
	\begin{aligned}
		T(h) &= \frac{16S(h)-S(2h)}{15} = \frac{2^4S(h)-S(2h)}{2^4-1}\\
		P(h) &= \frac{2^6T(h)-T(2h)}{2^6-1} = \frac{64T(h)-T(2h)}{63}
	\end{aligned}
\end{equation}

\subsubsection{Point 2}

\begin{equation}
	\begin{aligned}
		\begin{cases} 
			R_0(h) = D(h)\\
			R_i(h) = \frac{2^{2i}R_{(i-1)}(h) - R_{(i-1)}(2h)}{2^{2i}-1}
		\end{cases}
	\end{aligned}
\end{equation}

\subsubsection{Point 3}

\begin{equation}
	\begin{aligned}
		\begin{cases} 
			R_0(h) = D(h)\\
			R_i(h) = \frac{2^{2i}R_{(i-1)}(h) - R_{(i-1)}(\alpha h)}{2^{2i}-1}
		\end{cases}
	\end{aligned}
\end{equation}

// A CONFIRMER

\subsection{Question 3}

\subsubsection{Point 1}

// TO DO

\subsubsection{Point 2}

// TO DO

\subsubsection{Point 3}

// TO DO

\subsection{Question 4}

\subsubsection{Point 1}

// TO DO

\subsubsection{Point 2}

// TO DO

\subsubsection{Point 3}

// TO DO

\subsection{Question 5}

\subsubsection{Point 1}

// TO DO

\subsubsection{Point 2}

// TO DO